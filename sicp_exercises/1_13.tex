\documentclass{article}
\usepackage{graphicx}
\usepackage{amsmath}

\begin{document}

\title{SICP Problem 1.13}
\author{urmish}

\maketitle

\begin{abstract}
Prove that $Fib(n)$ is the closest integer to $\varphi^n/5.$
\end{abstract}

\section{Introduction}
Using hint in the original question, we approach the problem using proof by induction.
We can see 

\begin{equation}
    F(1) = \frac{\varphi^1 - \Psi^1}{\sqrt 5} = 1
\end{equation}

We assume the following to be true:

\begin{equation}
\label{assumption}
F(n) = \frac{\varphi^n - \Psi^n}{\sqrt 5}
\end{equation}

And prove the following statement:
\begin{equation}
F(n+1) = \frac{\varphi^{n+1} - \Psi^{n+1}}{\sqrt 5}
\end{equation}

The above equation can be rewritten as follows:

\begin{equation}
\begin{split}
\frac{\varphi^{n+1} - \Psi^{n+1}}{\sqrt 5} & = \frac{\varphi^n - \Psi^n}{\sqrt 5} + \frac{\varphi^{n-1} - \Psi^{n-1}}{\sqrt 5} \\
\varphi^{n+1} - \Psi^{n+1} & = \varphi^n - \Psi^n + \varphi^{n-1} - \Psi^{n-1} \\
\end{split}
\end{equation}

It should be sufficient to prove the above equation to prove the hint. \\

Note that:
\begin{equation}
\begin{split} 
\varphi + \Psi & = 1 \\
\varphi . \Psi & =  -1
\end{split}
\end{equation}

\begin{equation}
\begin{split}
  \varphi^{n+1} - \Psi^{n+1} & = (\varphi + \Psi) . (\varphi^n - \varphi^{n-1}.\Psi + \varphi^{n-2}.\Psi^2 - \varphi^{n-3}.\Psi^3 \dots - \varphi^2.\Psi^{n-2} + \varphi.\Psi^{n-1} - \Psi^n) \\
  & = (\varphi^n - \Psi^n) - (\varphi^{n-1}.\Psi - \varphi^{n-2}.\Psi^2 + \varphi^{n-3}.\Psi^3 - \dots + \varphi^2.\Psi^{n-2} - \varphi.\Psi^{n-1} ) \\
  & = (\varphi^n - \Psi^n) - (-\varphi^{n-2} + \varphi^{n-3}.\Psi - \varphi^{n-4}.\Psi^2 + \dots - \varphi^2\Psi^{n-4} + \varphi.\Psi^{n-3} - \Psi^{n-2} ) \\
  & = F(n) + (\varphi^{n-2} - \varphi^{n-3}.\Psi + \varphi^{n-4}.\Psi^2 - \dots + \varphi^2\Psi^{n-4} - \varphi.\Psi^{n-3} + \Psi^{n-2} ) \\
  & = F(n) + (\varphi + \Psi).(\varphi^{n-1} - \psi^{n-1}) \\
  & = F(n) + F(n-1)
\end{split}
\end{equation}

This proves the statement for $n+1$. You can see $\lvert\Psi\rvert < 1$. As $n$ increases $\lvert\Psi^n\rvert \to 0 $ proving $F(n) = \lfloor\varphi^n/\sqrt 5\rceil$


\end{document}
